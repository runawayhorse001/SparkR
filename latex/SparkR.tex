%% Generated by Sphinx.
\def\sphinxdocclass{report}
\documentclass[letterpaper,12pt,english]{sphinxmanual}
\ifdefined\pdfpxdimen
   \let\sphinxpxdimen\pdfpxdimen\else\newdimen\sphinxpxdimen
\fi \sphinxpxdimen=.75bp\relax

\PassOptionsToPackage{warn}{textcomp}
\usepackage[utf8]{inputenc}
\ifdefined\DeclareUnicodeCharacter
% support both utf8 and utf8x syntaxes
\edef\sphinxdqmaybe{\ifdefined\DeclareUnicodeCharacterAsOptional\string"\fi}
  \DeclareUnicodeCharacter{\sphinxdqmaybe00A0}{\nobreakspace}
  \DeclareUnicodeCharacter{\sphinxdqmaybe2500}{\sphinxunichar{2500}}
  \DeclareUnicodeCharacter{\sphinxdqmaybe2502}{\sphinxunichar{2502}}
  \DeclareUnicodeCharacter{\sphinxdqmaybe2514}{\sphinxunichar{2514}}
  \DeclareUnicodeCharacter{\sphinxdqmaybe251C}{\sphinxunichar{251C}}
  \DeclareUnicodeCharacter{\sphinxdqmaybe2572}{\textbackslash}
\fi
\usepackage{cmap}
\usepackage[T1]{fontenc}
\usepackage{amsmath,amssymb,amstext}
\usepackage{babel}
\usepackage{times}
\usepackage[Bjarne]{fncychap}
\usepackage{sphinx}

\fvset{fontsize=\small}
\usepackage{geometry}

% Include hyperref last.
\usepackage{hyperref}
% Fix anchor placement for figures with captions.
\usepackage{hypcap}% it must be loaded after hyperref.
% Set up styles of URL: it should be placed after hyperref.
\urlstyle{same}

\addto\captionsenglish{\renewcommand{\figurename}{Fig.}}
\addto\captionsenglish{\renewcommand{\tablename}{Table}}
\addto\captionsenglish{\renewcommand{\literalblockname}{Listing}}

\addto\captionsenglish{\renewcommand{\literalblockcontinuedname}{continued from previous page}}
\addto\captionsenglish{\renewcommand{\literalblockcontinuesname}{continues on next page}}
\addto\captionsenglish{\renewcommand{\sphinxnonalphabeticalgroupname}{Non-alphabetical}}
\addto\captionsenglish{\renewcommand{\sphinxsymbolsname}{Symbols}}
\addto\captionsenglish{\renewcommand{\sphinxnumbersname}{Numbers}}

\addto\extrasenglish{\def\pageautorefname{page}}

\setcounter{tocdepth}{2}

\usepackage{amsmath}
\usepackage{mathtools}
\usepackage{amsfonts}
\usepackage{amssymb}
\usepackage{dsfont}
\def\Z{\mathbb{Z}}
\def\R{\mathbb{R}}
\def\bX{\mathbf{X}}
\def\X{\mathbf{X}}
\def\By{\mathbf{y}}
\def\Bbeta{\boldsymbol{\beta}}
\def\bU{\mathbf{U}}
\def\bV{\mathbf{V}}
\def\V1{\mathds{1}}
\def\hU{\mathbf{\hat{U}}}
\def\hS{\mathbf{\hat{\Sigma}}}
\def\hV{\mathbf{\hat{V}}}
\def\E{\mathbf{E}}
\def\F{\mathbf{F}}
\def\x{\mathbf{x}}
\def\h{\mathbf{h}}
\def\v{\mathbf{v}}
\def\nv{\mathbf{v^{{f -}}}}
\def\nh{\mathbf{h^{{f -}}}}
\def\s{\mathbf{s}}
\def\b{\mathbf{b}}
\def\c{\mathbf{c}}
\def\W{\mathbf{W}}
\def\C{\mathbf{C}}
\def\P{\mathbf{P}}
\def\T{{\bf \mathcal T}}
\def\B{{\bf \mathcal B}}
\def\euler{\ e^{i\pi} + 1 = 0}


\title{ Learning Apache Spark with R}
\date{June 01, 2019}
\release{1.00}
\author{Wenqiang Feng, Wenhao Wang and Zhi Zheng}
\newcommand{\sphinxlogo}{\sphinxincludegraphics{logo.png}\par}
\renewcommand{\releasename}{Release}
\makeindex
\begin{document}

\pagestyle{empty}
\maketitle
\pagestyle{plain}
\sphinxtableofcontents
\pagestyle{normal}
\phantomsection\label{\detokenize{index::doc}}\phantomsection\label{\detokenize{index:index}}\begin{quote}

\begin{figure}[htbp]
\centering

\noindent\sphinxincludegraphics{{logo}.png}
\end{figure}
\end{quote}

Welcome to our Learning Apache Spark with R note! In this note, you will learn a wide array of concepts about SparkR in Data Mining, Text Mining, Machine Learning and Deep Learning. Some contents are from \sphinxcite{reference:feng2017}  The PDF version can be downloaded from \sphinxhref{SparkR.pdf}{HERE}.




\chapter{Preface}
\label{\detokenize{preface:preface}}\label{\detokenize{preface:id1}}\label{\detokenize{preface::doc}}

\section{About}
\label{\detokenize{preface:about}}

\subsection{About this note}
\label{\detokenize{preface:about-this-note}}
This is a shared repository for \sphinxhref{https://github.com/runawayhorse001/SparkR}{Learning Apache Spark with R}.
The PDF version can be downloaded from \sphinxhref{pyspark.pdf}{HERE}.
The first version was posted on Github in \sphinxhref{https://mingchen0919.github.io/learning-apache-spark/index.html}{ChenFeng} (\sphinxcite{reference:feng2017}).
This shared repository mainly contains the self-learning and
self-teaching notes from Wenqiang during his \sphinxhref{https://www.ima.umn.edu/2016-2017/SW1.23-3.10.17\#}{IMA Data Science
Fellowship}. The reader is referred to the repository \sphinxurl{https://github.com/runawayhorse001/LearningApacheSpark} for more
details about the \sphinxcode{\sphinxupquote{dataset}} and the \sphinxcode{\sphinxupquote{.ipynb}} files.

In this repository, I try to use the detailed demo code and
examples to show how to use each main functions. If you find
your work wasn’t cited in this note, please feel free to let
me know.

Although I am by no means an data mining programming and Big Data expert,
I decided that it would be useful for me to share what I learned
about PySpark programming in the form of easy tutorials with
detailed example. I hope those tutorials will be a valuable tool
for your studies.

The tutorials assume that the reader has a preliminary knowledge
of programing and Linux. And this document is generated automatically
by using \sphinxhref{http://sphinx.pocoo.org}{sphinx}.


\subsection{About the authors}
\label{\detokenize{preface:about-the-authors}}\begin{itemize}
\item {} 
\sphinxstylestrong{Wenqiang Feng}
\begin{itemize}
\item {} 
Data Scientist and PhD in Mathematics

\item {} 
University of Tennessee at Knoxville

\item {} 
Email: \sphinxhref{mailto:von198@gmail.com}{von198@gmail.com}

\end{itemize}

\sphinxstylestrong{Wenhao Wang}
\begin{itemize}
\item {} 
and PhD in

\item {} 
The University of Kansas

\item {} 
Email:

\end{itemize}

\sphinxstylestrong{James Zheng}
\begin{itemize}
\item {} 
Sr. Data Scientist and PhD in

\item {} 
University of Tennessee at Knoxville

\item {} 
Email:

\end{itemize}

\item {} 
\sphinxstylestrong{Biography}

Wenqiang Feng is Data Scientist within DST’s Applied Analytics Group. Dr. Feng’s responsibilities include providing DST clients with access to cutting-edge skills and technologies, including Big Data analytic solutions, advanced analytic and data enhancement techniques and modeling.

Dr. Feng has deep analytic expertise in data mining, analytic systems, machine learning algorithms, business intelligence, and applying Big Data tools to strategically solve industry problems in a cross-functional business. Before joining DST, Dr. Feng was an IMA Data Science Fellow at The Institute for Mathematics and its Applications (IMA) at the University of Minnesota. While there, he helped startup companies make marketing decisions based on deep predictive analytics.

Dr. Feng graduated from University of Tennessee, Knoxville, with Ph.D. in Computational Mathematics and Master’s degree in Statistics. He also holds Master’s degree in Computational Mathematics from Missouri University of Science and Technology (MST) and Master’s degree in Applied Mathematics from the University of Science and Technology of China (USTC).

\item {} 
\sphinxstylestrong{Declaration}

The work of Wenqiang Feng was supported by the IMA, while working at IMA. However, any opinion, finding, and conclusions or recommendations expressed in this material are those of the author and do not necessarily reflect the views of the IMA, UTK and DST.

\end{itemize}


\section{Motivation for this tutorial}
\label{\detokenize{preface:motivation-for-this-tutorial}}
I was motivated by the \sphinxhref{https://www.ima.umn.edu/2016-2017/SW1.23-3.10.17\#}{IMA Data Science Fellowship}
project to learn PySpark. After that I was impressed and attracted by the
PySpark. And I foud that:
\begin{enumerate}
\def\theenumi{\arabic{enumi}}
\def\labelenumi{\theenumi .}
\makeatletter\def\p@enumii{\p@enumi \theenumi .}\makeatother
\item {} 
It is no exaggeration to say that Spark is the most powerful
Bigdata tool.

\item {} 
However, I still found that learning Spark was a difficult
process. I have to Google it and identify which one is true.
And it was hard to find detailed examples which I can easily
learned the full process in one file.

\item {} 
Good sources are expensive for a graduate student.

\end{enumerate}


\section{Copyright notice and license info}
\label{\detokenize{preface:copyright-notice-and-license-info}}
This \sphinxhref{https://github.com/runawayhorse001/SparkR}{Learning Apache Spark with R} PDF file is supposed to be a free and living document, which is why its source is available online at \sphinxurl{https://runawayhorse001.github.io/LearningApacheSpark/pyspark.pdf}. But this document is licensed according to both \sphinxhref{https://github.com/runawayhorse001/LearningApacheSpark/blob/master/LICENSE}{MIT License} and  \sphinxhref{https://creativecommons.org/licenses/by-nc/2.0/legalcode}{Creative Commons Attribution-NonCommercial 2.0 Generic (CC BY-NC 2.0) License}.

\sphinxstylestrong{When you plan to use, copy, modify, merge, publish, distribute or sublicense, Please see the terms of those licenses for more details and give the corresponding credits to the author}.


\section{Acknowledgement}
\label{\detokenize{preface:acknowledgement}}
At here, I would like to thank Ming Chen, Jian Sun and Zhongbo Li at the
University of Tennessee at Knoxville for the valuable disscussion
and thank the generous anonymous authors for providing the detailed
solutions and source code on the internet. Without those help,
this repository would not have been possible to be made. Wenqiang
also would like to thank the \sphinxhref{https://www.ima.umn.edu/}{Institute for Mathematics and Its
Applications (IMA)} at \sphinxhref{https://twin-cities.umn.edu/}{University of Minnesota, Twin Cities}
for support during his IMA Data Scientist Fellow visit.

A special thank you goes to \sphinxhref{http://staffwww.dcs.shef.ac.uk/people/H.Lu/}{Dr. Haiping Lu}, Lecturer in Machine Learning
at Department of Computer Science, University of Sheffield, for recommending
and heavily using my tutorial in his teaching class and for the valuable
suggestions.


\section{Feedback and suggestions}
\label{\detokenize{preface:feedback-and-suggestions}}
Your comments and suggestions are highly appreciated. I am more
than happy to receive corrections, suggestions or feedbacks through
email (\sphinxhref{mailto:von198@gmail.com}{von198@gmail.com}) for improvements.


\chapter{Python Installation}
\label{\detokenize{install:python-installation}}\label{\detokenize{install:install}}\label{\detokenize{install::doc}}
\begin{sphinxadmonition}{note}{Note:}
This Chapter {\hyperref[\detokenize{install:install}]{\sphinxcrossref{\DUrole{std,std-ref}{Python Installation}}}} is for beginner.  If you have some \sphinxcode{\sphinxupquote{Python}} programming experience, you may skip this chapter.
\end{sphinxadmonition}

No matter what operator system is, I will strongly recommend you to install \sphinxcode{\sphinxupquote{Anaconda}} which contains \sphinxcode{\sphinxupquote{Python}}, \sphinxcode{\sphinxupquote{Jupyter}}, \sphinxcode{\sphinxupquote{spyder}}, \sphinxcode{\sphinxupquote{Numpy}}, \sphinxcode{\sphinxupquote{Scipy}}, \sphinxcode{\sphinxupquote{Numba}}, \sphinxcode{\sphinxupquote{pandas}}, \sphinxcode{\sphinxupquote{DASK}},
\sphinxcode{\sphinxupquote{Bokeh}}, \sphinxcode{\sphinxupquote{HoloViews}}, \sphinxcode{\sphinxupquote{Datashader}}, \sphinxcode{\sphinxupquote{matplotlib}}, \sphinxcode{\sphinxupquote{scikit-learn}}, \sphinxcode{\sphinxupquote{H2O.ai}}, \sphinxcode{\sphinxupquote{TensorFlow}}, \sphinxcode{\sphinxupquote{CONDA}} and more.

Download link: \sphinxurl{https://www.anaconda.com/distribution/}

\begin{figure}[htbp]
\centering

\noindent\sphinxincludegraphics{{anaconda}.png}
\end{figure}


\chapter{Main Reference}
\label{\detokenize{reference:main-reference}}\label{\detokenize{reference:reference}}\label{\detokenize{reference::doc}}
\begin{sphinxthebibliography}{Feng2017}
\bibitem[Feng2017]{reference:feng2017}
Wenqiang Feng. \sphinxhref{https://runawayhorse001.github.io/LearningApacheSpark/pyspark.pdf}{Learning Apache Spark with Python, 2017.}
\end{sphinxthebibliography}



\renewcommand{\indexname}{Index}
\printindex
\end{document}